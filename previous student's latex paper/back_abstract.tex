\begin{abstract}
A binary rewriter is a piece of software that accepts a binary executable program as input, and produces an improved executable as output. This paper describes the first technique in literature to decompile the input binary into an existing compiler's high-level intermediate representation (IR). The compiler's existing back-end is then used to generate the output binary from the IR. This enables the use of the rich set of compiler analysis passes available in mature compilers and also enables binary rewriters to perform complex high-level transformations not possible in existing binary rewriters.

We present our techniques to overcome various challenges which arise due to certain characteristics of binaries: the use of an explicitly addressed stack, the lack of function prototypes and the lack of symbols. Distributed binaries do not contain symbol-table or debugging information, hence, our methods make no use of any such metadata. We have integerated our techniques with a prototype binary rewriter called SecondWrite that uses LLVM, a widely-used compiler infrastructure, as IR, and rewrites x86 binaries. Our results show that SecondWrite accelerates un-optimized binaries by 45\% on average for our benchmarks, and further optimizes higly optimized binaries by 10\%, just by using standard optimizations present in LLVM. The robustness of our framework is displayed by rewriting binaries produced by two different compilers (gcc and Microsoft compiler).
%We also present the impact of our techniques on automatic parallelization to exemplify the benefits and applications of a binary rewriter using a compiler IR. 

\end{abstract}
