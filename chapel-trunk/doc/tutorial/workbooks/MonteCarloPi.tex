%% LyX 1.6.0 created this file.  For more info, see http://www.lyx.org/.
%% Do not edit unless you really know what you are doing.
\documentclass[11pt,english]{article}
\usepackage[T1]{fontenc}
\usepackage[latin9]{inputenc}
\usepackage{geometry}
\geometry{verbose,letterpaper}
\usepackage{amssymb}


%%%%%%%%%%%%%%%%%%%%%%%%%%%%%% LyX specific LaTeX commands.
\DeclareRobustCommand{\greektext}{%
  \fontencoding{LGR}\selectfont\def\encodingdefault{LGR}}
\DeclareRobustCommand{\textgreek}[1]{\leavevmode{\greektext #1}}
\DeclareFontEncoding{LGR}{}{}


%%%%%%%%%%%%%%%%%%%%%%%%%%%%%% Textclass specific LaTeX commands.
\newenvironment{lyxlist}[1]
{\begin{list}{}
{\settowidth{\labelwidth}{#1}
 \setlength{\leftmargin}{\labelwidth}
 \addtolength{\leftmargin}{\labelsep}
 \renewcommand{\makelabel}[1]{##1\hfil}}}
{\end{list}}

\usepackage{babel}

\begin{document}

\title{Estimating \textgreek{p} Using a Monte Carlo Method}

\maketitle
In this exercise, you will become familiar with Chapel's basic syntax,
including some of the parallel tasking features, while estimating
the value of $\pi$ using a Monte Carlo method.
\begin{enumerate}
\item As with any Monte Carlo method, you will need a pseudorandom number
generator. A simple one, taken from \cite{63042}, is a multiplicative
linear congruential generator defined by the following formula:\begin{equation}
x_{n+1}=ax_{n}\mathrm{mod}m,\label{eq:MLCG}\end{equation}
where the multiplier $a=16807$ and the modulus $m=2^{31}-1=2147483647$.
Note that multiplication and modulus are of the same operational order.
To get started with this exercise, you will implement this random
number generator as a function that takes an argument (corresponding
to $x_{n}$ in the formula above) and use it to compute and return
the next random number in the sequence. (If you are terrified of starting
with a blank screen, you can use RandomNumber0.chpl in the solutions
to the exercises as a starting point.) Be careful with how you do
the multiplication: the product $ax_{n}$ can be as large as $16807\times2147483646\thickapprox1.03\times2^{45}$.
\item To test your random number generator, you can write a main program
that invokes your random number generator 10000 times, with 1 as the
argument when you invoke it the very first time. Have your program
write out the last return value; if it writes out 1043618065, you
have implemented your random number generator correctly.
\item You can use your random number generator to simulate throwing darts
at random at a unit circle. Each dart will fall somewhere on a $1\times1$
grid. Therefore, you will need to modify your random number generator
to generate real numbers between 0 and 1 inclusive. One way to do
this would be to create another function that takes no arguments,
calls the one you just wrote, subtracts 1 from the return value (so
the value is now in the range of 0 and 2147483645 inclusive), and
then divides that by 2147483645 (using floating-point arithmetic).
Your new function can pass in a global variable to the one you wrote
before. See Chapter 9 of the Chapel Language Specification to learn about
conversions between variables of different types; an incorrect
conversion may lead your answer to always be 0.
\item Since you follow good software engineering practices, you will want
to encapsulate your code in a module. First, replace your random number
generator's argument (often referred to as the seed) with a (global)
config var that is initialized to 1. This will have the advantage
of allowing the initial value to be specified on the command line
when you run your program by using the {}``-- --<\emph{config-var-name}>=<\emph{value}>''
flag (without the double quotes), where \emph{config-var-name} is
the name of your config var, and \emph{value} is the initial value
you want to use (instead of 1). Next, to make things more convenient
later on, put your main program's code into a function that does not
take any arguments. Create a separate source file that uses the module
you just wrote and invokes what used to be the main program and is
now a function. You will need to use the {}``$\mathtt{--main-module}$
<\emph{module-name}>'' flag when you compile your source files together
to specify which module contains the main program. (Unless you specify
otherwise in your source code, the name of a module is taken from
its file name.)
\item In order to determine if a particular simulated dart hit inside the
unit circle, you can generate a random point $(x,y)$ in a $1\times1$
grid, and then calculate its distance from the origin using the formula
$d=\sqrt{x^{2}+y^{2}}$. If this distance $d\leq1$, then the dart
fell inside the unit circle. You can then calculate the proportion
of how many simulated darts fell inside the unit circle as compared
to the total number of simulated darts. This proportion corresponds
to the area within a $1\times1$ grid that is within the unit circle.
Now, the area of a unit circle is given by the formula $A=\pi r^{2}$,
where $r$ is the radius of the circle. Since in this case, the radius
$r=1$, the area is simply $\pi$. However, a $1\times1$ grid covers
only one of four quadrants in which the unit circle lies, so the proportion
of simulated darts that fall inside the unit circle within this grid
will have to be multiplied by 4. I.e.,\[
\pi\thickapprox4\times\frac{D_{\textrm{in}}}{D_{\textrm{total}}},\]
where $D_{\textrm{in}}$ is the number of simulated darts that fall
inside the unit circle within a $1\times1$ grid, and $D_{\textrm{total}}$
is the total number of simulated darts thrown.


In the separate source file you created above, write a program that
generates a random point (by invoking your random number generator
twice), and determines if it is inside the unit circle. Now, put a
loop around this, and keep a count of the total number of points generated,
as well as how many of those points fell inside the unit circle. Then,
print out the proportion of these counts multiplied by 4. Run your
program several times, each time with a different number of generated
points. (You may find it convenient to use a config const to control
the number of generated points from the command line you use to run
your program.) The accuracy of your estimate of $\pi$ should improve
as you increase the number of simulated darts.
\begin{lyxlist}{00.00.0000}
\item [{\emph{Optional}:}] Use the reduce operator with a square-bracketed
forall expression as its operand to count how many points are inside
the unit circle. Note that you can use a boolean expression as part
of the forall expression; the value of a boolean expression is 1 if
it evaluates to true, and 0 otherwise.
\end{lyxlist}
\item It would be great to improve the performance of your program by parallelizing
it. One way to do that would be to break up the counting of the points
among several tasks by using the coforall statement. This will require
a thread-safe random number generator. You can test if your random
number generator is thread safe by replacing the for statement in
what used to be the main program of your random number generator module
with a coforall statement. (Ignore the warning message about running
out of threads to run your program.) Since the computation for each
iteration is so short, it is unlikely you will run into any race conditions.
However, if you use a standard module called Time, and insert $\mathtt{sleep(1);}$
in the function that computes the next random number to delay its
computation by about a second, you will be much more likely to run
into a race condition, and the output of your program will very likely
be different.
\item In order to make your random number generator thread safe, you can
use a sync variable so that only one iteration of the coforall statement
can read and then modify the global variable your random number generator
uses: once a sync variable is read, no task (including the one that
just read it) can read it again until it has been written to. Now
that your random number generator is thread safe, replace the for
statement in your program to estimate $\pi$ with a coforall statement.
You may need to use a sync variable to ensure you get an accurate
count!
\item Unfortunately, this is not a very satisfactory way to parallelize
this program because the random number generator serializes execution.
There are several ways around this. One way is to have each iteration
use a different random number generator to create several points.
For the purposes of this exercise, you can create a new random number
generator by using different values for the multiplier $a$ in equation
\ref{eq:MLCG} above, such as 397204094 or 950706376. (See \cite{272709}.)
For each iteration, the count of points that lie inside the unit circle
can be accumulated in a separate variable, and these counts can be
added together at the end.
\item As an exercise for the reader, you can distribute the computation
to multiple locales, and have each locale execute one iteration of
the coforall statement. In order to avoid having each locale access
global arrays or variables, which would reside in locale 0, you can
encapsulate all the data that a locale will need into a class, and
then have each locale instantiate that class.
\end{enumerate}
\bibliographystyle{plain}
\bibliography{references}

\end{document}
