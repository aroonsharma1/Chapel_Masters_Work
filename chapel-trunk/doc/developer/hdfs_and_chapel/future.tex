\section{Future Work}\label{s:future}

Future work in this area has a couple different areas.

\subsection{Locality}

Right now, it is up to the programmer to create the ability to query where data is
stored as well how they wish to represent this data and make it viewable in Chapel.
However in the future it would be nice to have a locality API much like the current 
file IO API and therefore make it easier for the library writer to interface with Chapel 
level programs in such a way that they can leverage as much benefit from a different 
filesystem as possible. 

Also, it would be nice to have a way to deal only with data on a
certain locale when dealing with regular filesystems (such as LUSTRE, or other RAID 0
type filesystems) the goal being so you could do something along the lines of 
\begin{lstlisting}
// Connect to, and open a filesystem here
...
forall stripe in file.stripes by 3 {
  // Do something with every third stripe
}
\end{lstlisting}
Which gives the ability to easily reason about and use concurrency in RAID
0 type filesystems.

\subsection{Replication of Files}
Right now due to the way HDFS is structured, when {\tt hdfsOpen} is called, it creates a
file per locale. This is not ideal, but at least at this point the only way to be
able to handle all cases in HDFS since we do not know beforehand where the various
blocks of the file will reside.

On other filesystems, this could be different, and therefore the possibility to
minimize the number of files created becomes a possiblity. As well as this also might
allow us to represent this information in a more compact and meaningful way. 

Along with these options is the option to make this file creation and replication
lazy in the sense that files will only be created on a locale iff a file is needed
on that locale. Otherwise, that locale will not get a file. This coupled with
caching the file once we've opened it on a locale would also offer a speed
improvement.

