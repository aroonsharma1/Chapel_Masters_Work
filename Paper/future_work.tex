\section{Future Work}\label{sec:future_work}

As presented, the modulo unrolling optimization can be improved upon in a few ways to achieve even better performance in practice. First, there is currently no limit on the number of array elements that an aggregate message may contain. For applications with extremely large data sets, buffers containing remote data elements may become too large and exceed the memory budget of a particular locale. This may slow down other programs running on the system. To solve this, the modulo unrolling optimization should perform strip mining where the aggregate message is broken down into smaller sections if it contains too many elements in order to conserve memory. 

The modulo unrolling optimization currently performs both aggregate reading of remote data elements before the loop and aggregate writing of remote data elements after the loop no matter what the loop body consists of. It is conceivable that some of the yielded elements during zippered iteration will not be read or written to at all during the loop. An improvement to the optimization would be to avoid prefetching elements that are not read in the loop body and to avoid writing back elements that are not written to in the loop body. 
