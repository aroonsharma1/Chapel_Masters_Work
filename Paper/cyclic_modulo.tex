\section{Cyclic Distribution with Modulo Unrolling}\label{sec:cyclic_modulo}

Modulo unrolling is implemented in the follower iterator of the Cyclic distribution. Based on the semantics of parallel zippered iteration, the leader iterator will divide up the iterations of the loop across the locales of the machine according to the first item in the zippering. This could mean that some portions of work will not be local to where the computation is taking place. The follower iterator in the Cyclic distribution recognizes whether or not its chunk of work is local or remote. If remote, all of the remote array elements are brought to the present locale in a local buffer using one \texttt{chpl\_comm\_gets} call. Finally, elements of the local buffer are now yielded back to the loop header. A loop body may modify the elements that are yielded to it via zippered iteration. To account for this, the follower iterator compares the element before it was yielded to the element after it was yielded. If any of the elements in the follower's chunk of work were modified, the entire local buffer is stored back to the remote local via one \texttt{chpl\_comm\_puts} call. 